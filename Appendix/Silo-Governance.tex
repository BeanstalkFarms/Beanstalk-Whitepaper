\documentclass[class=article, crop=false]{standalone}
\usepackage[subpreambles=true]{standalone}
\usepackage{import}
\usepackage{amsmath}
\usepackage{enumitem} % Format list spacing 

\begin{document}

\subsection{Former Governance}
The following has been removed from \hyperlink{subsection.5.5}{Section 5.5 Governance} as part of the updates to reflect Beanstalk's current permissioned governance system and is left here to contribute to the discussion around a future permissionless governance system. 

The submitter of a \term{BIP} automatically votes in favor of the \term{BIP}, cannot rescind their vote, and cannot have less than $K^{\text{min}}$ of total outstanding \term{Stalk} after an interaction with the \term{Silo}, until the end of the \term{Voting Period}.

When a \term{BIP} passes or has a two-thirds majority, it must be manually committed to the Ethereum blockchain. To encourage prompt commitment of \term{BIPs} even during periods of congestion on the Ethereum network while minimizing cost, the award for successful commitment starts at 100 Beans and compounds 1\% every additional six seconds that elapse past the end of its \term{Voting Period} ($E_{\text{BIP}}$) for 1,800 seconds.

The award for successfully committing an approved \term{BIP} ($a^q$), such that $a^q \in \{j \times 10^{-6} \mid j \in \mathbb{Z}^{+} \}$, with a given timestamp of commitment ($E_q$) and $E_{\text{BIP}}$ is:
$$a^q = 100 \times 1.01^{\text{min}\left\{\left\lfloor\frac{E_q - E_{\text{BIP}}}{6}\right\rfloor,\ 300\right\}}$$
To minimize the cost of calculating $a^q$, Beanstalk uses a binomial estimation with a margin of error of less than 0.05\%. When a \term{BIP} is committed with a two-thirds supermajority before the end of its \term{Voting Period}, $a^q = 100$.

\end{document}